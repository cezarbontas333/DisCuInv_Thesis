\chapter{Introduction}
\label{chap:introduction}

\myLettrine{S}{olar} energy and its conversion methods to electrical energy have been heavily explored \cite{pena2014robust, de2023sliding, 10.3389/fenrg.2024.1498514, systematiclcl2015}over the past decades as technological advancements in cell structure, computing power and topologies have improved the viability of this method for energy production.
It is much more important in today's climate as the energy input has significantly increased as to satisfy the energy consumption from industrial and personal activities.
This has lead to further demand for solar energy conversion infrastructure, especially in areas where the reliance on the electrical output of the grid is suboptimal.
A popular choice in recent years is establishing a microgrid architecture \cite{pena2014robust}, where each household has an array of \gls{pv} panels that provides energy, and the surplus can be stored in battery cells or send to compensate the electrical load in the grid.
These applications are rather small in total power output, so there's an incentive for low-cost, easy and scalable implementations to be researched.

There are a lot of methods used for solar conversion, however many of them involve complex mathematical algorithms or expensive components, which may be a hindrance to the average household consumer\cite{rashid2013power}. 
From hybrid usage of current conduction modes \cite{morroni2009adaptive}, maximum power point tracking, or sliding mode control \cite{de2023sliding}, many of these methods revolve around controlling the current flow through \gls{pwm} signal generation which drives active switching circuits, which in power applications most of them are transistors.
Each type of transistor has different properties, from switching time, power consumption and heat dissipation, that affects the operation of the circuits involved in the conversion.
The biggest issue with the methods mentioned and the standard ones used in current applications are that hardware signal generation blocks tend to require complex algorithms \cite{reznik2013lcl} to create the desired command, thus making the implementation hard to both implement on low-cost hardware and to further tune should the plant change its parameters.

I believe that cheap, easy to manufacture solutions are important in promoting green energy to a wider audience, especially households and hobbyists, that would want to improve their quality of life.
In this thesis I will provide a simpler method for current control of inverters, both in hardware architecture and regulator implementation, based around a voltage source inverter with an LCL filter which acts as a constant power source.
The controller algorithm used in conjunction is a simple \gls{pi} regulator, which should more appropriate on smaller scale applications, with power targets known.
I will present both \gls{ccm} and \gls{dcm}, why \gls{dcm} has great potential in the described application, a mathematical model used for simulating it, and to design a prototype \gls{pcb} around these concepts.
Finally, I will interpret the results given by the simulation and try to pinpoint ways in which the proposed solution can be improved.