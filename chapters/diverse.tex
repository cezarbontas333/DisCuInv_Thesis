\chapter{Despre plagiat}
\label{chap:div}


În România, legea drepturilor de autor este \textbf{Legea nr. 8/1996} completată de \textbf{Legea nr. 285 din 23 iunie 2004} şi \textbf{Ordonanţa de urgenţă 123 din 1 septembrie 2005}.

\myLettrine{C}{onform} Dicţionarului Explicativ al Limbii Române:

\blockquote{PLAGIA: A-şi însuşi, a copia total sau parţial ideile, operele etc. cuiva, prezentându-le drept creaţii personale; a comite un furt literar, artistic sau ştiinţific.}

În contextul lucrărilor ştiinţifice, plagiatul reprezintă utilizarea ideilor, tehnologiilor, rezultatelor sau textelor altor persoane, omițând referirea lucrării originale.

În cazul secţiunilor de text citate din alte opere (exemplu mai sus), se recomandă încadrarea între ghilimele a textului, cu mențiunea sursei. Nu se recomandă preluări de text mai lungi de 2-3 rânduri.

În cazul prezentării unor idei, teorii, fapte statistice etc., care nu ţin de cultura generală și sunt preluate din alte opere, se recomandă re-povestirea prin prisma înțelegerii proprii și citarea sursei în prima frază. De exemplu: {\em Algoritmul X a fost propus în [3] și constă în următoarea secvență de operații.}

Modul cel mai simplu de evitare a plagiatului este formularea personală a lucrării, cu menționarea surselor acolo unde este cazul. Parafrazarea cu schimbarea câtorva cuvinte nu este suficientă.

Toate lucrările de diplomă sunt supuse unei verificări antiplagiat cu un program specializat. Rezultatele verificării sunt interpretate de un cadru didactic. Nu se poate vorbi despre un procent admisibil de similaritate cu texte existente. Copierea a 2-3 paragrafe fără a cita sursa, chiar și cu modificarea unor cuvinte, poate duce la scăderea notei la lucrarea de diplomă. Copierea unor porțiuni mari de text, de ordinul paginilor, poate duce la neprimirea în examenul de diplomă și eventual chiar la pedepse mai aspre. Desigur, este mai gravă prezentarea ca rezultate personale a unor rezultate obținute de alții (deci copierea "contribuțiilor proprii") decât copierea unor "noțiuni teoretice", dar ambele sunt la fel de interzise.


